\chapter{Introduction}

In the beginning was MVC. The original MVC pattern was created by Trygve
Reenskaug while at Xerox PARC in 1978 and 79\footnotemark[1] to solve
the problem of allowing users to view data in multipule ways. MVC found
its way in the GUI and from there onto the web. At each point though, it
changed slightly. When used in as a solution to the problem of GUIs its
primary purpose was to separate a programs logic from its presentation,
not to give the user multiple interfaces. Once it made its way into the
web the model was replaced with another pattened called the Object
Relational Model (ORM).

The point of this book is to provide an overview of webframeworks.
A webframework is a collection of code that provides a framework for
building web applications. Some well known webframeworks are \emph{Ruby
on Rails}, \emph{Django}, and \emph{CakePHP}.

\footnotetext[1]{If you'd like to learn more about the original MVC
pattern see: http://heim.ifi.uio.no/~trygver/}

\section{Types of Webframeworks}

There are two major categories of webframeworks:

\begin{itemize}
    \item Request \& Response
    \item Declarative
\end{itemize}

Request \& Response frameworks such as Django, and Flask thrive on a
request object being passed throughout the framework. At some point a
response object is returned back to the server and send to the connected
client.

Declarative, or 'Automagical' frameworks like Rails, and CakePHP, work
by disguising requests and responses. All framework users have to do is
write views that return HTML, or a template, which the framework knows
how to convert into a response. The users generally never directly touch
or worry about things like middleware or authorization, as the framework
provides specific ways of interacting, or adding these things. For
example a global 'user' object that any view can access.
other 
like middleware
