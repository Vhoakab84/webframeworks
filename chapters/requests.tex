\chapter{Requests}

Requests are the one of the first deviations from the classic MVC
framework. Websites are navigated through a series of requests to a
server. These requests are then passed through to an application that
knows what to do when someone asks for \emph{/home.html}

These requests don't fit well at all into the MVC framework. They could
be thought of as controllers, but since they pass a user onto a
controller that won't work.

\section{Types}

There are three ways of setting up request dispatching:

\begin{itemize}
    \item Single Route Map
    \item Attached Routes
    \item Class based Routes
\end{itemize}

\subsection{Single Route Map}

Projects that use a single route map, most notably Django, take the
easiest approach to routing. All the routes for the web application live
in a single file, and each is associated with a view. The view code is
not imported into the file, as this might create circular dependencies
if some view implemented a dynamic lookup. Instead the associated view
to a route is listed as a module string, like 'webapp.blog.add_post'.
The association is then handled by a seperate piece of the framework
that uses the string to import the modules during the process of
initialization. In Django, this would be the WSGI handler.

\subsection{Attached Routes}

Frameworks like Flask, Bottle, and Klein takes a simpler approach.
Instead of having an extra file that holds all associations of routes to
views, views are wrapped in a route decorator.

This has several benefits. The association between a specific route and
view is immediatly obvious; there is no need to open the routes file,
look up the route and corresponding view, then open the view file.
Because the routes are implemented as middleware, other middleware
decorator can easily be added and applied to views.

The drawback to using decorators to associate routes is decorators
themselves. In Python decorators are applied and parsed at import time,
not execution. Therefore any work that might assume an object exists
that a decorator instantiated, or any view that might change ways it is
routed at runtime can not work.

\subsection{Class based Routes}

Using classes to define routes is used in frameworks such as KohanaPHP.
Instead of listing routes in a file, or appying them to a view. A
classed view's methods define the routes that are available. Most of
these classed would implement a \emph{index} method, which relates to
`/'. Other methods like \emph{blog} or \emph{build} would have
corresponding routes already created for them at `/blog' and `/build'.
Any arguments the methods take would become \textbf{GET} or
\textbf{POST} parameters to the route.
